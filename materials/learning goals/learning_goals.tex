\documentclass{article}[12pt]

\usepackage{amsmath, amssymb}
\usepackage{eucal}
\usepackage[dvipsnames]{xcolor}
\usepackage{hyperref}
\hypersetup{
    colorlinks = true,
    linkcolor = RoyalPurple,
    urlcolor = YellowOrange
}
\usepackage{twemojis}


\title{MAT605 Logic and Foundations with Haskell}
\author{Marius Furter}

\begin{document}

\setlength{\twemojiDefaultHeight}{1em}

\maketitle

\section{Resources}
The video material for the course may be found either on \href{https://uzh.mediaspace.cast.switch.ch/channel/23FS%2BMAT605%2BLogic%2Band%2BFoundations%2Bwith%2BHaskell/32019}{SWITCHcast} or on \href{https://www.youtube.com/watch?v=0HImO-me_sg&list=PLd8NbPjkXPliojM8YMN3z3o9--zXwti8Z}{YouTube}. In both places the videos contain chapter headings to help you navigate the material.

\section{Learning goals}
The following is a summary of the learning goals of the course by week. Each learning goal for the theory will be marked with one of the following symbols:
\begin{description}
    \item[\twemoji{slightly smiling face} (essentials)] These points are essential knowledge for passing the course. If you complete all of them, you should have no trouble passing the exam.
    \item[\twemoji{thinking face} (food for thought)] These points are more advanced and might require more work to understand. They are not required in order to get a passing grade, but some of these topics might appear on the exam.
    \item[\twemoji{exploding head} (exploding head)] These points are completely optional. They are either difficult, tedious, or things I included for your culture. They will not appear on the exam, but they might help you understand the other material better.
\end{description}

Aside from the learning goals, I will also provide some questions for you to contemplate.

\subsection*{Week 1}

\paragraph{Haskell 1 :: Setup} This video covers how to get Haskell installed on your computer. I recommend using \href{https://www.haskell.org/ghcup/}{GHC} together with \href{https://code.visualstudio.com}{Visual Studio Code}. In order to make the two work together, you need to let GHCup install all the stuff it asks you to, in particular you need the Haskell language server. In VSCode you will need to install the \texttt{Haskell} and \texttt{Haskell Syntax Highlighting} extensions.

\paragraph{Course Intro} This video introduces the course. Watching it is optional.

\paragraph{Logic 1 :: Introduction} This video introduces the logic portion of the course. It doesn't contain any essential material, but it sets up some ideas that will be helpful later. I would recommend watching it.

\emph{Question: Think about some mathematical statements of your favorite area. What does it mean for such statements to be `true'? How would you translate this to formulas in a logical system?}

\paragraph{Logic 2 :: Naive Propositional Logic} This video covers the basics of propositional logic in an informal manner. We will see all of this in more detail in a few weeks. You probably have already encountered all of this material in your studies, but I included it so as to not assume any prior knowledge. If you feel like you satisfy the learning goals, feel free to skip this video. If you are unclear on some points, I would recommend skimming the video using the chapter feature.
\begin{itemize}
    \item[\twemoji{slightly smiling face}] Understand the \emph{syntax} of propositional logic. Be able to recognize and write well-formed formulas.
    \item[\twemoji{slightly smiling face}] Know the \emph{truth table definitions} for the usual logical connectives ($\neg$, $\wedge$, $\vee$, $\rightarrow$, $\leftrightarrow$). Be able to calculate the truth value of propositional logc formulas.
    \item[\twemoji{slightly smiling face}] Know the definition of \emph{logical validity} and \emph{logical equivalence} for propositional formulas. Be able to demonstrate these using truth tables.
    \item[\twemoji{slightly smiling face}] Be able to prove new logical equivalences from old ones by \emph{substitution}.
\end{itemize}

\subsection*{Week 2}

\paragraph{Haskell 2 :: Basic Operations } This video covers basic operations in Haskell such as arithmetic, comparison, lists, ranges, list comprehension, and simple functions. It follows \href{http://learnyouahaskell.com/starting-out}{chapter 2} of the ``Learn you a Haskell for great good'' tutorial.
\begin{itemize}
    \item[\twemoji{slightly smiling face}] Be able to do \emph{basic arithmetic} in Haskell
    \item[\twemoji{slightly smiling face}] Be able to \emph{compare} objects
    \item[\twemoji{slightly smiling face}] Be able to define \emph{lists and ranges} and understand their limitations
    \item[\twemoji{slightly smiling face}] Understand the syntax for \emph{list comprehension}
    \item[\twemoji{slightly smiling face}] Know how to write \emph{basic functions}
    \item[\twemoji{slightly smiling face}] Know the difference between \emph{prefix and infix functions}. Be able to convert each into the other
    \item[\twemoji{slightly smiling face}] Be able to define \emph{tuples} and understand their limitations 
\end{itemize}

\paragraph{Haskell 3 :: Types and Typeclasses} This video covers the type system in Haskell. It follows \href{http://learnyouahaskell.com/types-and-typeclasses}{chapter 3} of the ``Learn you a Haskell for great good'' tutorial.
\begin{itemize}
    \item[\twemoji{slightly smiling face}] Know the \emph{basic types} listed in the \href{http://learnyouahaskell.com/types-and-typeclasses}{online tutorial}. Be able to create objects of each of them.
    \item[\twemoji{slightly smiling face}] Know the syntax for declaring the \emph{types of function} with one or more arguments
    \item[\twemoji{slightly smiling face}] Know how to use \emph{type variables and typeclasses} in type declarations for functions
    \item[\twemoji{slightly smiling face}] Be familiar with the most common \emph{typeclasses} (Eq, Ord, Num, Integral, Floating)
\end{itemize}

\paragraph{Logic 3 :: Naive First Order Logic} This video covers first order logic informally based on examples. You probably already are familiar with much of the material in this video. I would recommend watching it if you haven't thought a lot about the structure of first order logic formulas before. You will then be better prepared for the formal definitions that will come later.
\begin{itemize}
    \item[\twemoji{slightly smiling face}] Be able to parse well-formed first order logic formulas into a \emph{parsing tree}
    \item[\twemoji{slightly smiling face}] Be able to \emph{translate} between precise mathematical statements and first order logic formulas
    \item[\twemoji{slightly smiling face}] Be able to convert between \emph{restricted quantifiers} and unrestricted ones
\end{itemize}

\subsection*{Week 3}

\paragraph{Haskell 4 :: Functions} This video covers how to define functions using pattern matching and guards. It also discusses \texttt{where}, \texttt{let}, and \texttt{case}.
\begin{itemize}
    \item[\twemoji{slightly smiling face}] Know the \emph{syntax for pattern matching} numbers, tuples, and lists.
    \item[\twemoji{slightly smiling face}] Know the \emph{syntax} for defining functions using \emph{guards}.
    \item[\twemoji{slightly smiling face}] Know the \emph{syntax} for \texttt{where} and \texttt{let}.
\end{itemize}

\paragraph{Logic 4 :: Informal Proof Theory} This video covers all the rules necessary to prove first order logic formulas. The main goal is for you to know and be able to apply these rules. Since most of you will already have quite some experience in proving things, this material should seem familiar. However, you might not have explicitly given the rules much thought. For additional examples and exercises, I recommend looking at Velleman's book \href{https://www.cambridge.org/highereducation/books/how-to-prove-it/6D2965D625C6836CD4A785A2C843B3DA#overview}{``How to prove it''}.

\begin{itemize}
    \item[\twemoji{slightly smiling face}] Know the proof rules for all first order logic connectives. I have summarized the rules \href{https://github.com/MariusFurter/MAT605-haskell-FS23/blob/main/materials/proof%20rules/proof_rules.pdf}{here}.
    \item[\twemoji{thinking face}] Be able to \emph{write basic proofs} using the rules.
    \item[\twemoji{slightly smiling face}] Be able to recognize the \emph{logical equivalences} involving quantifiers presented at the end of the video.
\end{itemize}

\subsection*{Week 4}

\paragraph{Haskell 5 :: Implementing Logical Functions} In this video we implement our own version of the type Bool along with functions acting on Bool.
\begin{itemize}
    \item[\twemoji{slightly smiling face}] Be able to write functions on the level of the translation functions and the logical operators \texttt{not}, \texttt{\&\&}, $ \mathtt{\vert \vert}$.
    \item[\twemoji{slightly smiling face}] Be able to understand the recursive definitions for \texttt{and}, \texttt{or}, \texttt{elem}, \texttt{all}, \texttt{any}, and \texttt{filter}.
    \item[\twemoji{thinking face}] Be able to implement functions on the level of \texttt{and}, \texttt{or} using recursion.
\end{itemize}

\paragraph{Logic 5 :: Natural Deduction} This video covers the formal proof system called natural deduction. It roughly follows chapter 2 of Chiswell and Hodges' ``Mathematical Logic''. The rules are basically the same as those presented in last week's video on informal proof theory.

\begin{itemize}
    \item[\twemoji{slightly smiling face}] Know what a \emph{derivation} is.
    \item[\twemoji{slightly smiling face}] Know what a \emph{sequent} is.
    \item[\twemoji{slightly smiling face}] Be able to \emph{recognize the natural deduction rules} for each connective and be able to identify if a derivation is correct.
    \item[\twemoji{slightly smiling face}] Be able to \emph{recognize whether a given sequent rule follows} from a given natural deduction rule.
    \item[\twemoji{thinking face}] Be able to \emph{derive sequent rules} from natural deduction rules.
    \item[\twemoji{thinking face}] Be able to \emph{write derivations} that prove simple sequents (ca.~2-4 rule applications).
    \item[\twemoji{exploding head}] Think about proof by contradiction and the law of the excluded middle. In what situations in life would you accept an answer based on `proof by contradiction'. Can you find situations where a double negation $\neg \neg \varphi$ does not seem equivalent to $\varphi$. If you would create your own mathematical world, what level of proof would you require to call statements `true'?
\end{itemize}

\section{Example exam questions}
The written exam will consist of two types of questions that will cover both theory and the basic syntax of Haskell. The first 2/3 of the exam will test your knowledge of the essential learning goals. These questions will not require much thought. If you get about 80\% of these correct, you will pass the exam. If you get all of them right, that would be about a $4.7$. The remaining 1/3 will consist of questions that either require more thinking or come from the food for thought learning goals. To get a $6$ you need to score around 88\% of points overall. I will adjust the grading scale if it turns out to be too strict. You will also be allowed to write down reasons for your answers to the multiple choice question if you are unsure about a question so that I can assign partial points if your reasoning is partially correct.

\subsection{Example essential questions}
The learning goal: `Be able to recognize well-formed formulas of propositional logic' could translate into the exam question:

\emph{Which of the following formulas are well-formed formulas of propositional logic?}
\begin{itemize}
    \item[$\square$] $\forall x \exists y ((p_1(x) \wedge p_2(x) ) \rightarrow p_3(y))$
    \item[$\square$] $ ((p_1 \vee p_2) \wedge p_3) $
    \item[$\square$] $ (p_1 \rightarrow p_2 \rightarrow p_3)$
    \item[$\square$] $ ((p_1 \neg p_2) \vee p_3 )$
\end{itemize}

The learning goals: `Know how to write basic functions' and `Know the syntax for pattern matching tuples' might translate to the question:

\emph{Which of the following pieces of code define valid functions in Haskell?}
\begin{itemize}
    \item[$\square$] \verb|fun x y = x+y|
    \item[$\square$] \verb|Fun x = x+x|
    \item[$\square$] \verb|fun(x,y) = x+y|
    \item[$\square$] \verb|fun (x,y) = x+y|
\end{itemize}

The learning goal `Know what a sequent is' might translate to the question

\emph{What is the correct definition of a sequent?}
\begin{itemize}
    \item[$\square$] A sequent is an expression $\Gamma \vdash \varphi$. It is correct, if whenever all the assumptions in $\Gamma$ are true, then so is $\varphi$.
    \item[] \item[$\square$]  A sequent is a type of formal proof of the form $\Gamma \vdash \varphi$ that uses only the sequent rules of natural deduction.
    \item[$\square$] A sequent is an expression $\Gamma \vdash \varphi$. It is correct, if there exists a derivation with conclusion $\varphi$ using only assumptions occurring in $\Gamma$.
    \item[$\square$]  A sequent is a rule of the form $\Gamma \vdash \varphi$ that may be used in natural deduction derivation to prove $\varphi$ from $\Gamma$.
\end{itemize}

\subsection{Example thinking questions}
The learning goals: `Be able to translate between precise mathematical statements and first order logic formulas' and `Be able to translate between restricted and unrestricted quantification' might translate to the question:

\emph{Which of the following first order logic formulas capture the statement: `For all $x$ in $A$ there exists $y$ in $A$ satisfying $x < y$' ?}
\begin{itemize}
    \item[$\square$] $ \forall x \exists y ( x \in A \wedge (y \in A \wedge \phi < \psi)) $
    \item[$\square$] $ \exists y \in A \forall x \in A (x < y) $ 
    \item[$\square$] $ \forall x \in A \exists y \in A (x < y) $
    \item[$\square$] $ \forall x \exists y (x \in A \rightarrow (y \in A \wedge x < y)) $
\end{itemize}

The learning goal: `Be able to derive new logical equivalence from old ones by substitution' might translate into the following question:

\emph{Consider the following logical equivalences: 
$$\Phi \rightarrow \Psi \cong (\neg \Phi) \vee \Psi, \qquad
 \neg (\neg \Phi) \cong \Phi,  $$
 $$\neg (\Phi \vee \Psi) \cong (\neg \Phi) \wedge (\neg \Psi), \qquad \neg (\Phi \wedge \Psi) \cong (\neg \Phi) \vee (\neg \Psi).$$ 
 Use these to show that
$$ (\neg (\Phi \rightarrow \Psi)) \rightarrow (\neg \Psi) \cong ((\neg \Phi) \vee \Psi) \vee (\neg \Psi) $$ 
What does this tell you about this formula?
}

The learning goal: `Be able to write derivations that prove simple sequents' might translate to:

\emph{Provide a derivation that proves $\{ \phi, \psi \} \vdash (\phi \vee \tau) \wedge \psi$}

\end{document}
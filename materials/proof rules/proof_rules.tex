\documentclass{article}[12pt]

\usepackage{amsmath, amssymb}
\usepackage{eucal}
\usepackage[dvipsnames]{xcolor}
\usepackage{hyperref}
\hypersetup{
    colorlinks = true,
    linkcolor = RoyalPurple,
    urlcolor = YellowOrange
}
\usepackage{twemojis}

\usepackage{array}
\def\arraystretch{1.5}

\title{Summary of informal proof rules}
\author{Marius Furter}

\begin{document}

\section*{Summary of informal proof rules}

\begin{tabular}{ c | m{14em} | m{14em} }
    & \textbf{Introduction Rule} & \textbf{Elimination Rule} \\ 
    \hline \hline
    $\rightarrow$ &
    To prove $\Phi \rightarrow \Psi$, assume $\Phi$ and prove $\Psi$. &
    Given $\Phi$ and $\Phi \rightarrow \Psi$, conclude $\Psi$. \\
    \hline
    $\leftrightarrow$ &
    To prove $\Phi \leftrightarrow \Psi$, prove both $\Phi \rightarrow \Psi$ and $\Psi \rightarrow \Phi$. &
    Given $\Phi \leftrightarrow \Psi$ and $\Phi$, conclude $\Psi$.

    Given $\Phi \leftrightarrow \Psi$ and $\Psi$, conclude $\Phi$. \\
    \hline
    $\neg$ &
    To prove $\neg \Phi$, assume $\Phi$ and derive a contradiction. &
    Given $\neg \Phi$ and $\Phi$, conclude $\Psi$. \\
    \hline
    $\wedge $ &
    To prove $\Phi \wedge \Psi$, prove both $\Phi$ and $\Psi$. &
    Given $\Phi \wedge \Psi$, conclude $\Phi$.
    
    Given $\Phi \wedge \Psi$, conclude $\Psi$.\\
    \hline
    $\vee$ &
    To prove $\Phi \vee \Psi$, prove at least one of $\Phi$ or $\Psi$. &
    To use $\Phi \vee \Psi$ to prove $\Gamma$, first assume $\Phi$ and prove $\Gamma$. Then separately assume $\Psi$ and again prove $\Gamma$. \\
    \hline
    $\forall$ &
    To prove $\forall x \Phi(x)$, pick an arbitrary $x_0$ and prove $\Phi(x_0)$. &
    Given $\forall x \Phi(x)$, conclude $\Phi(x_0)$ for any $x_0$ of your choice. \\
    \hline
    $\exists$ &
    To prove $\exists x \Phi(x)$, specify an $x_0$ satisfying $\Phi(x_0)$. &
    Given $\exists x \Phi(x)$, let $x_0$ be an element satisfying $\Phi(x_0)$. \\
    \hline
\end{tabular}

\subsubsection*{Proof by contradiction}
Use the $(\neg)$-introduction rule on $\neg \Phi$ to conclude $\neg(\neg \Phi) \equiv \Phi$.

\subsubsection*{Proof by cases}
Add the tautology $\Phi \vee \neg \Phi$ to your givens and use the $(\vee)$-elimination rule to distinguish cases.

\subsubsection*{Modifying givens and proof goal}
One can always replace the givens or the proof goal with logically equivalent statements. One may always add established true statements to the givens.

\end{document}